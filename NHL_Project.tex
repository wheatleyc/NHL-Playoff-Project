% Options for packages loaded elsewhere
\PassOptionsToPackage{unicode}{hyperref}
\PassOptionsToPackage{hyphens}{url}
%
\documentclass[
]{article}
\usepackage{amsmath,amssymb}
\usepackage{lmodern}
\usepackage{iftex}
\ifPDFTeX
  \usepackage[T1]{fontenc}
  \usepackage[utf8]{inputenc}
  \usepackage{textcomp} % provide euro and other symbols
\else % if luatex or xetex
  \usepackage{unicode-math}
  \defaultfontfeatures{Scale=MatchLowercase}
  \defaultfontfeatures[\rmfamily]{Ligatures=TeX,Scale=1}
\fi
% Use upquote if available, for straight quotes in verbatim environments
\IfFileExists{upquote.sty}{\usepackage{upquote}}{}
\IfFileExists{microtype.sty}{% use microtype if available
  \usepackage[]{microtype}
  \UseMicrotypeSet[protrusion]{basicmath} % disable protrusion for tt fonts
}{}
\makeatletter
\@ifundefined{KOMAClassName}{% if non-KOMA class
  \IfFileExists{parskip.sty}{%
    \usepackage{parskip}
  }{% else
    \setlength{\parindent}{0pt}
    \setlength{\parskip}{6pt plus 2pt minus 1pt}}
}{% if KOMA class
  \KOMAoptions{parskip=half}}
\makeatother
\usepackage{xcolor}
\usepackage[margin=1in]{geometry}
\usepackage{graphicx}
\makeatletter
\def\maxwidth{\ifdim\Gin@nat@width>\linewidth\linewidth\else\Gin@nat@width\fi}
\def\maxheight{\ifdim\Gin@nat@height>\textheight\textheight\else\Gin@nat@height\fi}
\makeatother
% Scale images if necessary, so that they will not overflow the page
% margins by default, and it is still possible to overwrite the defaults
% using explicit options in \includegraphics[width, height, ...]{}
\setkeys{Gin}{width=\maxwidth,height=\maxheight,keepaspectratio}
% Set default figure placement to htbp
\makeatletter
\def\fps@figure{htbp}
\makeatother
\setlength{\emergencystretch}{3em} % prevent overfull lines
\providecommand{\tightlist}{%
  \setlength{\itemsep}{0pt}\setlength{\parskip}{0pt}}
\setcounter{secnumdepth}{-\maxdimen} % remove section numbering
\usepackage{booktabs}
\usepackage{longtable}
\usepackage{array}
\usepackage{multirow}
\usepackage{wrapfig}
\usepackage{float}
\usepackage{colortbl}
\usepackage{pdflscape}
\usepackage{tabu}
\usepackage{threeparttable}
\usepackage{threeparttablex}
\usepackage[normalem]{ulem}
\usepackage{makecell}
\usepackage{xcolor}
\ifLuaTeX
  \usepackage{selnolig}  % disable illegal ligatures
\fi
\IfFileExists{bookmark.sty}{\usepackage{bookmark}}{\usepackage{hyperref}}
\IfFileExists{xurl.sty}{\usepackage{xurl}}{} % add URL line breaks if available
\urlstyle{same} % disable monospaced font for URLs
\hypersetup{
  pdftitle={Data Mining and Machine Learning Final Project},
  pdfauthor={Daniil Deych, Alex Mykietyn, Cameron Wheatley},
  hidelinks,
  pdfcreator={LaTeX via pandoc}}

\title{Data Mining and Machine Learning Final Project}
\author{Daniil Deych, Alex Mykietyn, Cameron Wheatley}
\date{2023-04-24}

\begin{document}
\maketitle

\hypertarget{what-team-relative-regular-season-statistics-can-tell-us-about-nhl-playoff-team-performance}{%
\section{What team-relative regular season statistics can tell us about
NHL playoff team
performance?}\label{what-team-relative-regular-season-statistics-can-tell-us-about-nhl-playoff-team-performance}}

\hypertarget{i.-introduction-and-background}{%
\subsection{I. Introduction and
Background}\label{i.-introduction-and-background}}

As team and player statistics have garnered a more prominent role in
sports, attempts at predicting performance has become a go to area for
Las Vegas bookies and sports fans in general. Getting into any debate
about how ``good'' your team will be will undoubtedly incur some version
of statistical performance. Be it your gut or a fancy statistical model,
those numbers fuel the desire to predict the future. We are no
different.

With this project we are hoping to build a predictive model which NHL
team will win a given 7 game playoff series using the teams' regular
season statistics. Using the difference between teams' statistics in we
calculate the home team's advantage (or disadvantage) in each category.

Historically, the primary Machine Learning tool used in predicting
sports performance has been neural networks (Weissbock el al., 2013),
here we are looking to test a variety of other Machine Learning
models.The end goal of all these methods being to find which model will
provide the highest percentage of correct winners on the testing set.

A typical regular season hockey game consists of three 20-min periods.
During the period, each team puts out five players and a goalie onto
ice, and they attempt to win the game by putting a puck into the net of
the other team. If by the end of the three periods the game is tied, the
game goes into a 5 minute ``sudden death'' over time, after which, if
still tied, the game goes into a shootout. In the shootout, each team
takes turns by sending out one player to score a penalty shot on the
opposing team's goalie. After 3 attempts, the team with most penalty
shots scored wins the match and the losing team gets attributed an
Overtime Loss (OTL). These overtime rules were instituted part way
through our data set in 2006, previously a tie would be assigned if no
one scored in the overtime period. If during the game, either team
commits a foul, the other team receives a Power Play for a
pre-determined amount of time (typically 2 mins), during which the
fouling player gets sent off for that amount of time and the other team
plays with an extra player on the ice, this situation often results in a
goal.

\hypertarget{ii.-data}{%
\subsection{II. Data}\label{ii.-data}}

NHL's website has a copious amounts of data going as far back as 1917,
but the game has changed dramatically since then, so we had to be
discerning about which years to look at for our data. We decided to
choose the most recent 20 seasons that played all the 82 mandated games,
which makes the earliest season we accounted for to be 1998.

As a side note, we excluded several shortened seasons that took place
between 1998 and now. COVID pandemic shortened the 2020-2021 and
2019-2020 season, while player strikes lead to lockouts for the 2012-13
and 2004-05 season.

To create our data set, we used the raw data from NHL's website and
created our unique data set. For each year, we isolated all the playoff
match ups that year, and created a separate row for each match up,
designated by Home/Away team, the rest of the row lists the regular
season difference-statistics between Home/Away teams of the given
matchup. (See Appendix below for more details). It is this difference in
regular season stats that will fuel our predictive models.

The variables of interest attempt to describe team performance in
various situations. Many variables track break down the wins, losses and
win percentage of teams based on the margin of victory in regular season
games. We also have statistics relating to performance by period and the
propensity of a team to come from behind or blow leads. Considering the
number of goals scored on Power Plays, we have statistics measuring team
performance in these situations as well as the frequency they take
penalties and draw penalties from the other team.

\hypertarget{iii.-method}{%
\subsection{III. Method}\label{iii.-method}}

The main challenge with our approach is our relatively small data
subset. With only 20 seasons and 300 playoff series against more than 60
variables, we do not have sufficient rows of data to work with the data
set directly. To account for that we elected to use step wise selection
and lasso approach to reduce the number of variables. In both cases we
are concerned about the degrees of freedom in our model and so we limit
variable selection to a maximum of 10 variables given that we will only
be training on 200 observations.

We will select variables using a lasso regression and a stepwise
selection process. We will then plug the selected variables into probit,
ordered probit, and random forests models. Given that some of our
statistics are highly correlated, we will also use Principle Component
Analysis to summarize the data set and plug this into a probit
model.Given the nature of our small data set we attempt to give
ourselves less variance in our testing set we hold out (100
observations) from our training data. To attempt to avoid over fitting
our models we test potential parameter values on resamples of the
training set and select the model that performs best on these training
set resamples . We then test the accuracy for each model by calculating
the absolute improvement and lift against our null model that Home team
(which is equivalent to being a higher seed) always wins.

\begin{verbatim}
## Loading required package: Matrix
\end{verbatim}

\begin{verbatim}
## -- Attaching core tidyverse packages ------------------------ tidyverse 2.0.0 --
## v dplyr     1.1.0     v readr     2.1.4
## v forcats   1.0.0     v stringr   1.5.0
## v ggplot2   3.4.1     v tibble    3.1.8
## v lubridate 1.9.2     v tidyr     1.3.0
## v purrr     1.0.1     
## -- Conflicts ------------------------------------------ tidyverse_conflicts() --
## x tidyr::expand() masks Matrix::expand()
## x dplyr::filter() masks stats::filter()
## x dplyr::lag()    masks stats::lag()
## x tidyr::pack()   masks Matrix::pack()
## x tidyr::unpack() masks Matrix::unpack()
## i Use the ]8;;http://conflicted.r-lib.org/conflicted package]8;; to force all conflicts to become errors
## Registered S3 method overwritten by 'mosaic':
##   method                           from   
##   fortify.SpatialPolygonsDataFrame ggplot2
## 
## 
## The 'mosaic' package masks several functions from core packages in order to add 
## additional features.  The original behavior of these functions should not be affected by this.
## 
## 
## Attaching package: 'mosaic'
## 
## 
## The following objects are masked from 'package:dplyr':
## 
##     count, do, tally
## 
## 
## The following object is masked from 'package:purrr':
## 
##     cross
## 
## 
## The following object is masked from 'package:ggplot2':
## 
##     stat
## 
## 
## The following object is masked from 'package:Matrix':
## 
##     mean
## 
## 
## The following objects are masked from 'package:stats':
## 
##     binom.test, cor, cor.test, cov, fivenum, IQR, median, prop.test,
##     quantile, sd, t.test, var
## 
## 
## The following objects are masked from 'package:base':
## 
##     max, mean, min, prod, range, sample, sum
## 
## 
## 
## Attaching package: 'foreach'
## 
## 
## The following objects are masked from 'package:purrr':
## 
##     accumulate, when
## 
## 
## 
## Attaching package: 'modelr'
## 
## 
## The following object is masked from 'package:mosaic':
## 
##     resample
## 
## 
## The following object is masked from 'package:ggformula':
## 
##     na.warn
## 
## 
## Loading required package: Hmisc
## 
## 
## Attaching package: 'Hmisc'
## 
## 
## The following objects are masked from 'package:dplyr':
## 
##     src, summarize
## 
## 
## The following objects are masked from 'package:base':
## 
##     format.pval, units
## 
## 
## 
## Attaching package: 'caret'
## 
## 
## The following object is masked from 'package:mosaic':
## 
##     dotPlot
## 
## 
## The following object is masked from 'package:purrr':
## 
##     lift
## 
## 
## 
## Attaching package: 'MASS'
## 
## 
## The following object is masked from 'package:dplyr':
## 
##     select
## 
## 
## 
## Attaching package: 'psych'
## 
## 
## The following object is masked from 'package:Hmisc':
## 
##     describe
## 
## 
## The following objects are masked from 'package:mosaic':
## 
##     logit, rescale
## 
## 
## The following objects are masked from 'package:ggplot2':
## 
##     %+%, alpha
## 
## 
## Loaded lars 1.3
## 
## 
## 
## Attaching package: 'lars'
## 
## 
## The following object is masked from 'package:psych':
## 
##     error.bars
## 
## 
## Rows: 300 Columns: 56
## -- Column specification --------------------------------------------------------
## Delimiter: ","
## chr  (2): Away, Home
## dbl (54): W.L, Win_Margin, Year, Point_Differential, RW, GF.GP, GA.GP, Total...
## 
## i Use `spec()` to retrieve the full column specification for this data.
## i Specify the column types or set `show_col_types = FALSE` to quiet this message.
\end{verbatim}

\hypertarget{step-wise-selection}{%
\subsubsection{Step-wise Selection}\label{step-wise-selection}}

\begin{verbatim}
## Reordering variables and trying again:
\end{verbatim}

Step-wise selection variables are - face-off win percentage (FOW\%),
penalty kill percentage (Net.PK), percentage of games won byc 2 goals
(win..2goal.game), number of penalties drawng against the other team
(Pen.Drawn.60), percentage of games won by more than 3 goals
(win..3.goal.game), goals against in second period (GA.in.p2),
percentage of games won while leading in period 2 (win..lead.2p),
percentage of games won while leading in period 1 (win..lead.1p)

\hypertarget{lasso-selection}{%
\subsubsection{Lasso selection}\label{lasso-selection}}

Lasso selected variables are total goal differential
(total\_goal\_differential), goals against in period 2 (GA.in.P2),
percentage of games after scoring first (W..SF), net power play kill
percentage (Net.PK.), percentage of games won while leading in period 2
(win..lead.2p), percentage of games won by more than 3 goals
(win..3.goal.game), number of penalties drawn against the other team
(Pen.Drawn.60), shots per game differential (shots\_differential),
regulation wins (RW).

\hypertarget{using-lasso-selected-variables}{%
\subsection{Using lasso selected
variables}\label{using-lasso-selected-variables}}

\hypertarget{probit-model}{%
\subsubsection{Probit Model}\label{probit-model}}

\begin{verbatim}
##    yhat
## y    0  1
##   0 11 27
##   1  9 53
\end{verbatim}

\begin{verbatim}
## [1] "Probit Model Accuracy"
\end{verbatim}

\begin{verbatim}
## [1] 0.64
\end{verbatim}

\begin{verbatim}
## [1] "Null Model Accuracy"
\end{verbatim}

\begin{verbatim}
## [1] 0.62
\end{verbatim}

\hypertarget{ordered-probit-model}{%
\subsubsection{Ordered Probit Model}\label{ordered-probit-model}}

\begin{verbatim}
##     Ordered_Probit_Lasso_Pred
##      -4 -3 -2 -1  1  2  3  4
##   -4  0  0  0  2  0  3  0  0
##   -3  0  0  0  6  0  3  0  0
##   -2  0  0  0 10  0  7  0  0
##   -1  0  0  0  2  0  5  0  0
##   1   0  0  0  3  0  9  2  0
##   2   0  0  0  4  0 16  3  0
##   3   0  0  0  6  0  7  2  0
##   4   0  0  0  2  0  7  1  0
\end{verbatim}

\begin{verbatim}
## [1] "Ordered Probit Model Accuracy"
\end{verbatim}

\begin{verbatim}
## [1] 0.2
\end{verbatim}

\begin{verbatim}
## [1] "Null Model Accuracy"
\end{verbatim}

\begin{verbatim}
## [1] 0.23
\end{verbatim}

\hypertarget{logit-model}{%
\subsubsection{Logit Model}\label{logit-model}}

\begin{verbatim}
##    yhat
## y    0  1
##   0 11 27
##   1  9 53
\end{verbatim}

\begin{verbatim}
## [1] "LogitModel Accuracy"
\end{verbatim}

\begin{verbatim}
## [1] 0.64
\end{verbatim}

\begin{verbatim}
## [1] "Null Model Accuracy"
\end{verbatim}

\begin{verbatim}
## [1] 0.62
\end{verbatim}

\hypertarget{random-forest}{%
\subsubsection{Random Forest}\label{random-forest}}

\begin{verbatim}
##    yhat
## y    0  1
##   0 12 26
##   1  9 53
\end{verbatim}

\begin{verbatim}
##    yhat
## y    0  1
##   0 12 26
##   1  9 53
\end{verbatim}

\begin{verbatim}
## [1] "Forest Model Accuracy"
\end{verbatim}

\begin{verbatim}
## [1] 0.65
\end{verbatim}

\begin{verbatim}
## [1] "Null Model Accuracy"
\end{verbatim}

\begin{verbatim}
## [1] 0.62
\end{verbatim}

\hypertarget{step-wise-selection-1}{%
\subsection{Step-wise Selection}\label{step-wise-selection-1}}

\hypertarget{probit-model-1}{%
\subsubsection{Probit Model}\label{probit-model-1}}

\begin{verbatim}
##    yhat
## y    0  1
##   0 16 22
##   1 10 52
\end{verbatim}

\begin{verbatim}
## [1] "Probit Model Accuracy"
\end{verbatim}

\begin{verbatim}
## [1] 0.68
\end{verbatim}

\begin{verbatim}
## [1] "Null Model Accuracy"
\end{verbatim}

\begin{verbatim}
## [1] 0.62
\end{verbatim}

\hypertarget{ordered-probit-model-1}{%
\subsubsection{Ordered Probit Model}\label{ordered-probit-model-1}}

\begin{verbatim}
##     Ordered_Probit_Step_Pred
##      -4 -3 -2 -1  1  2  3  4
##   -4  0  0  0  4  0  1  0  0
##   -3  0  0  0  4  0  5  0  0
##   -2  1  0  0  8  0  7  1  0
##   -1  0  0  0  5  0  2  0  0
##   1   0  0  0  5  0  7  2  0
##   2   0  0  0  4  0 11  8  0
##   3   0  0  0  5  0  6  4  0
##   4   0  0  0  4  0  5  1  0
\end{verbatim}

\begin{verbatim}
## [1] "Ordered Probit Model Accuracy"
\end{verbatim}

\begin{verbatim}
## [1] 0.2
\end{verbatim}

\begin{verbatim}
## [1] "Null Model Accuracy"
\end{verbatim}

\begin{verbatim}
## [1] 0.23
\end{verbatim}

\hypertarget{logit-model-1}{%
\subsubsection{Logit Model}\label{logit-model-1}}

\begin{verbatim}
##    yhat
## y    0  1
##   0 16 22
##   1 10 52
\end{verbatim}

\begin{verbatim}
## [1] "Logit Model Accuracy"
\end{verbatim}

\begin{verbatim}
## [1] 0.68
\end{verbatim}

\begin{verbatim}
## [1] "Null Model Accuracy"
\end{verbatim}

\begin{verbatim}
## [1] 0.62
\end{verbatim}

\hypertarget{random-forest-1}{%
\subsubsection{Random Forest}\label{random-forest-1}}

\begin{verbatim}
##    yhat
## y    0  1
##   0 13 25
##   1  8 54
\end{verbatim}

\begin{verbatim}
## [1] "Forest Model Accuracy"
\end{verbatim}

\begin{verbatim}
## [1] 0.67
\end{verbatim}

\begin{verbatim}
## [1] "Null Model Accuracy"
\end{verbatim}

\begin{verbatim}
## [1] 0.62
\end{verbatim}

\hypertarget{pca}{%
\subsection{PCA}\label{pca}}

\begin{verbatim}
##    yhat
## y    0  1
##   0 14 24
##   1 11 51
\end{verbatim}

\begin{verbatim}
## [1] "PCA Model Accuracy"
\end{verbatim}

\begin{verbatim}
## [1] 0.65
\end{verbatim}

\begin{verbatim}
## [1] "Null Model Accuracy"
\end{verbatim}

\begin{verbatim}
## [1] 0.62
\end{verbatim}

\hypertarget{iv.-results}{%
\subsection{IV. Results}\label{iv.-results}}

\hypertarget{pca-1}{%
\subsubsection{PCA:}\label{pca-1}}

PCA Absolute Improvement = 3\%, Lift = 1.05

\#\#\#Lasso selected variables

Probit Model: Absolute Improvement = 2\%, Lift = 1.03

Ordered Probit Model: Absolute Improvement = -3\%, Lift = 0.87

Logit Model: Absolute Improvement = 6\%, Lift = 1.09

Random Forest: Absolute Improvement = 3\%, Lift = 1.05

\#\#\#Step-wise selected variables

Probit Model: Absolute Improvement = 6\%, Lift = 1.10

Ordered Probit Model: Absolute Improvement = -3\%, Lift = 0.87

Logit Model: Absolute Improvement = 6\%, Lift = 1.10

Random Forest: Absolute Improvement = 5\%, Lift = 1.08

\hypertarget{v.-conclusion}{%
\subsection{V. Conclusion}\label{v.-conclusion}}

In the end none of our models showed consistent improvement over the
base model that predicts the higher seed (Home team) to be the winner of
the match up. The improvement and lift vary significantly depending on
the seed and are not reliable. Upon discussion we came up with several
potential reasons for that.

Firstly, our data set is not expansive enough. With 60 variables and
only 300 observation that is not rigorous enough to run good train/test
splits, as especially with a smaller test set the variance in the data
is bound to be captured in the model. We attempted to account for
overfitting in the training set by resampling, but that technique did
not show much improvement either.

Another possible explanation is that our data set did not include some
of the potential confounding variables. Since NHL data set that we used
simply calculates the season average statistics, and the team
performance is more relevant to how the team ends the season, our data
does not account for that kind of heterogeneity.

A very common practice for teams that are playoff bound is to improve
their roster by bringing on high quality players later in the season,
whose performance is not likely to show up on the team's season long
statistics. To account for that a good statistic to add would have been
a share of the salary cap that is being used by the team, as the team
goes into the playoffs. The assumption there being that teams with no
salary space under the cap are likely to have the better players than
the ones that do.

The next confounder that we could not account for is the health of the
team. If an important player on the team was out for most of the season
due to a serious injury that would dramatically reduce their regular
season stats, which our models would predict to reflect in their playoff
runs. The reverse of that scenario would be true as well, an effect of
high impact player that going down right before the beginning of the
playoffs would not show up in any of our models.

To summarize, step-wise selection seemed to perform better across all of
our models, but the variance of our attempts to measure performance but
this in question. More data would and needed to properly run these
algorithms with lower variation in the number of home team wins in the
training and test splits.

\end{document}
